\cvitem
    {Researcher / Engineer}
    {2021 -- today (6 months)}
    {Gradiant}
    {Vigo, Pontevedra}
\begin{itemize}
  \small
  \item Usage of Deep Learning in the context of Industry 4.0, particularly for
    inference and control of processes.
  \item Currently in a quest for MLOps.
\end{itemize}

\cvitem
    {Research Technician}
    {2019 -- 2021 (1 year \& 10 months)}
    {Centro Singular de Investigación en Tecnoloxías Intelixentes}
    {Santiago de Compostela, A Coruña}
\begin{itemize}
    \small
    \item Took part in an Industry 4.0 project.
    \item Developed Deep Learning models for inference of industrial
      processes, using dense, recurrent and convolutional networks.
    \item Hyperparameter optimization.
    \item Feature selection algorithms.
    \item Data analysis.
\end{itemize}


\cvitem
    {ADAS Technician}
    {2018 -- 2019 (1 year \& 2 months)}
    {Centro Tecnológico de Automoción de Galicia}
    {O Porriño, Pontevedra}
\begin{itemize}
    \small
    \item Research in autonomous driving.
    \item Developed agorithms for LiDAR sensor data processing.
    \item Implemented algorithm performance surpassed sensor's native software.
    \item Design of algorithms for multi-object tracking.
\end{itemize}

\cvitem
    {Junior Researcher}
    {2015 -- 2016 (1 year)}
    {Universidad de Zaragoza}
    {Zaragoza}
\begin{itemize}
    \small
    \item Matlab developer for a joint bussiness-University project.
    \item Implemented dimensionality reduction techniques in the context of a
      termo-mechanical problem.
    \item There was a significant improvement in computing time with respect to the
      reference model (using ANSYS Mechanical™).
    \item Implemented import/export code for interoperability between ANSYS™ and
      XDMF3 formats.
    \item Research in model dimensionality reduction for scalar and vector problems.
\end{itemize}

