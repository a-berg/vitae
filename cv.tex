\documentclass[a4]{article}

% \usepackage[margin=15mm]{geometry}
\usepackage{graphicx}
\usepackage[fixed]{fontawesome5}
\usepackage{fontspec}
\usepackage{fancyhdr}
\usepackage{enumitem}
\usepackage{titlesec}
\usepackage[ocgcolorlinks]{hyperref}
\usepackage[spanish]{babel}
\usepackage{longtable}
\usepackage{enumitem}
\usepackage{tabularx}

\setlength{\parindent}{0pt}
% small hack for tables
\newcommand*{\toprule}[1][]{}
\newcommand*{\bottomrule}[1][]{}
\newcommand*{\midrule}[1][]{}
%% \setlength{\LTleft}{1em}
%% \setlength{\LTpre}{0pt}
%% \setlength{\LTpost}{0pt}

\hypersetup{
    colorlinks=true,
    linkcolor=cyan,
}
\titleformat{name=\section,numberless}[block]
  {\titlerule\addvspace{2pt}\sffamily\fontsize{9}{11}}
  {}{0pt}{\addfontfeature{LetterSpace=4.0}\uppercase}

%% \titlespacing\subsection{0pt}{2pt plus 4pt minus 2pt}{0pt}
% \titleformat{name=\subsection,numberless}[block]
%   {\titlerule\addvspace{2pt}\sffamily\fontsize{7}{9}}
%   {}{0pt}{\addfontfeature{LetterSpace=4.0}\textit}

\setmainfont[
    Numbers={Monospaced,Uppercase},
    BoldFont=Vollkorn SemiBold]{Vollkorn Regular}
\setsansfont{Open Sans}
% \setmonofont{Iosevka SS04}
\newfontfamily{\myName}{Open Sans}[
    LetterSpace=10, WordSpace=3
]

\setitemize{itemsep=-1pt}

\newcommand{\cvitem}[4]{
    \noindent
    \textbf{\large #1}  \hfill {\small #2} \newline
    \textit{#3}         \hfill #4
}

\newcommand{\projitem}[3]{
    \noindent
    \textbf{\large #1} con {\small #2} \vspace{2pt}\newline
    #3
    \vspace{8pt}
}

\newcommand{\pseudosection}[1]{
  % A subsection, but not really
    \noindent
    \textit{\sffamily{#1}}
    
    \vspace{6pt}
}

\renewcommand{\labelitemi}{$\vcenter{\hbox{\textbullet}}$}
\def\Cpp{{C\nolinebreak[4]\hspace{-.05em}\raisebox{.1ex}{++}}}

\fancyhead[R]{\sffamily\footnotesize Updated on \today}
\renewcommand{\headrulewidth}{0pt}
\pagestyle{fancy}

\usepackage[margin=15mm]{geometry}
\usepackage{graphicx}
\usepackage[fixed]{fontawesome5}
\usepackage{fontspec}
\usepackage{fancyhdr}
\usepackage{enumitem}
\usepackage{titlesec}
\usepackage[ocgcolorlinks]{hyperref}
\usepackage[spanish]{babel}
\usepackage{longtable}
\usepackage{enumitem}
\usepackage{tabularx}

\setlength{\parindent}{0pt}
% small hack for tables
\newcommand*{\toprule}[1][]{}
\newcommand*{\bottomrule}[1][]{}
\newcommand*{\midrule}[1][]{}
%% \setlength{\LTleft}{1em}
%% \setlength{\LTpre}{0pt}
%% \setlength{\LTpost}{0pt}

\hypersetup{
    colorlinks=true,
    linkcolor=cyan,
}
\titleformat{name=\section,numberless}[block]
  {\titlerule\addvspace{2pt}\sffamily\fontsize{9}{11}}
  {}{0pt}{\addfontfeature{LetterSpace=4.0}\uppercase}

%% \titlespacing\subsection{0pt}{2pt plus 4pt minus 2pt}{0pt}
% \titleformat{name=\subsection,numberless}[block]
%   {\titlerule\addvspace{2pt}\sffamily\fontsize{7}{9}}
%   {}{0pt}{\addfontfeature{LetterSpace=4.0}\textit}

\setmainfont[
    Numbers={Monospaced,Uppercase},
    BoldFont=Vollkorn SemiBold]{Vollkorn Regular}
\setsansfont{Open Sans}
% \setmonofont{Iosevka SS04}
\newfontfamily{\myName}{Open Sans}[
    LetterSpace=10, WordSpace=3
]

\setitemize{itemsep=-1pt}

\newcommand{\cvitem}[4]{
    \noindent
    \textbf{\large #1}  \hfill {\small #2} \newline
    \textit{#3}         \hfill #4
}

\newcommand{\projitem}[3]{
    \noindent
    \textbf{\large #1} con {\small #2} \vspace{2pt}\newline
    #3
    \vspace{8pt}
}

\newcommand{\pseudosection}[1]{
  % A subsection, but not really
    \noindent
    \textit{\sffamily{#1}}
    
    \vspace{6pt}
}

\renewcommand{\labelitemi}{$\vcenter{\hbox{\textbullet}}$}
\def\Cpp{{C\nolinebreak[4]\hspace{-.05em}\raisebox{.1ex}{++}}}


\fancyhead[R]{\sffamily\footnotesize Actualizado el \today}
\renewcommand{\headrulewidth}{0pt}
\pagestyle{fancy}
\begin{document}

\begin{flushleft}
    \begin{minipage}[t]{0.2\textwidth}
        \vspace{0pt}
        \includegraphics{./docs/PortraitCropped.jpg}
    \end{minipage}
    % \hspace{3mm}
    \textsf{
    \begin{minipage}[t]{0.65\textwidth}
        \vspace{-3pt}
        {\myName\LARGE\bfseries\uppercase{Adrián Berges
Enfedaque}}\\[8pt]
        \faEnvelope\ \    aberges@outlook.com\\[5pt]
        \faPhone\ \      (+34) 690 236 818\\[5pt]
        \faMapMarker*\ \ \kern-0.05em Santiago de Compostela, A
Coruña\\[5pt]
        % \faGlobe\ \ \href{https://a-berg.github.io/philambdapi/}{blog}\\[5pt]
        \faLinkedinIn\ \  \href{http://www.linkedin.com/in/adrianberges/}{/in/adrianberges}\\[5pt]
        \faGithub\ \      \href{http://www.github.com/a-berg}{a-berg}
    \end{minipage}
    }
\end{flushleft}

\section*{Experiencia}
%% This is a template for pandoc usage
\cvitem
    {Data Scientist}
    {2021 -- now (3 years \& 2 months)}
    {Gradiant}
    {Vigo, Pontevedra}
\begin{enumerate}[label=\textbullet, itemsep=-1pt, rightmargin=0.5cm]
  \item Solved ML problems with time series and tabular data, using Deep
Learning, Gradient Boosting, or Reinforcement Learning (among
others).\item Performed extensive data analysis in
Python.\item Experienced in MLOps and ML Engineering, such as experiment
tracking, model deployment and versioning. I also was an early member of
the then-emerging AI\&Data Ops team.\item Work with cloud technologies,
particularly Kubernetes, Airflow and Spark.\item Recently, had the
opportunity to manage a team of three developers in a project management
role.\item Mentored junior data scientists at my company, emphasizing
software design principles and effective use of ML and MLOps
tools.\item Organized and led coding-dojos on diverse topics such as an
introduction to pola.rs, Julia, or functional programming techniques in
Python.\item Delivered a guest lecture to third-year Artificial
Intelligence BSc students at the University of Santiago de Compostela,
providing an introduction to industry applications in Industry 4.0.
\end{enumerate}

\vspace{12pt}%% This is a template for pandoc usage
\cvitem
    {Research Technician}
    {2019 -- 2021 (1 year \& 9 months)}
    {Centro Singular de Investigación en Tecnoloxías Intelixentes}
    {Santiago de Compostela, A Coruña}
\begin{enumerate}[label=\textbullet, itemsep=-1pt, rightmargin=0.5cm]
  \item Utilized neural networks (with Keras and Tensorflow) to model
industrial processes as a mid-level developer.\item Participated in the
implementation of prescriptive analytics to control and automate the one
part of the process.\item Experience in mentoring junior team members,
uploading and maintaining models in client's machines and Hyperparameter
tuning with a HPC cluster.\item Proficient in various neural network
architectures and feature selection algorithms.
\end{enumerate}

\vspace{12pt}%% This is a template for pandoc usage
\cvitem
    {ADAS Technician}
    {2018 -- 2019 (1 year \& 2 months)}
    {Centro Tecnológico de Automoción de Galicia}
    {O Porriño, Pontevedra}
\begin{enumerate}[label=\textbullet, itemsep=-1pt, rightmargin=0.5cm]
  \item Developed custom LiDAR point cloud segmentation
algorithm.\item Successfully improved sensor
performance.\item Entry-level knowledge in multi-object tracking and
sensor fusion using Kalman filters.\item Gained junior-level
\Cpp programming expertise in the context of 3D Computer Vision
algorithms.
\end{enumerate}

\vspace{12pt}%% This is a template for pandoc usage
\cvitem
    {Junior Researcher}
    {2015 -- 2016 (1 year)}
    {Universidad de Zaragoza}
    {Zaragoza}
\begin{enumerate}[label=\textbullet, itemsep=-1pt, rightmargin=0.5cm]
  \item Implemented an algorithm in Matlab using dimensionality
reduction techniques for a thermal-mechanical problem in a novel
research setting as a junior.\item The implemented solution was orders
of magnitude faster than commercial software.\item Worked under the
guidance of mentors.\item Gained strong skills in Matlab, and
understanding of how dimensionality reduction techniques can be applied
to an industrial setting.
\end{enumerate}

\vspace{12pt}

\section*{Educación}

\cvitem
    {MSc Industrial Mathematics}
    {2016 -- 2018 }
    {Universidade de Santiago de Compostela}
    {Santiago de Compostela}
\vspace{10pt}\\\cvitem
    {BSc Mechanical Engineering}
    {2010 -- 2015 }
    {Universidad de Zaragoza}
    {Zaragoza}
\vspace{10pt}\\

\pseudosection{Educación no reglada}

\cvitem
    {Probabilistic Deep Learning with Tensorflow 2}
    {2020 -- 2020 (5 weeks)}
    {Imperial College London via Coursera}
    {Online}
\vspace{10pt}\\\cvitem
    {Machine Learning Engineering for Production (MLOps)}
    {2021 -- 2022 (16 weeks)}
    {DeepLearning.AI via Coursera}
    {Online}
\vspace{10pt}\\

\section*{Habilidades}

\subsection*{Languages}

%% this is actually genius: lists can be "pasted" using [<sep>].
\begin{tabularx}{\dimexpr\textwidth-0.5cm\relax}[t]{lX}
    Spanish & Native\\
    English & CEFRL C1 (CAE)\\
  \end{tabularx}

\subsection*{Soft skills}

%% this is actually genius: lists can be "pasted" using [<sep>].
\begin{tabularx}{\dimexpr\textwidth-0.5cm\relax}[t]{lX}
    Communication & I'm accustomed to convey information to
stakeholders. Can explain complex topics in a simple manner.\\
    Teamwork & Integrated my job in cross-functional teams.\\
    Adaptability & Have worked in several domains and project settings,
quickly building productive relationships with team members and
clients.\\
    Attention to detail & Very focused towards identifying and
correcting errors and inconsistencies.\\
    Methodic & Tend to approach tasks and projects in a systematic and
organized manner.\\
  \end{tabularx}

\subsection*{Technical skills}

%% this is actually genius: lists can be "pasted" using [<sep>].
\begin{tabularx}{\dimexpr\textwidth-0.5cm\relax}[t]{lX}
    Programming languages & Python, SQL, \Cpp, Matlab\\
    Data Science, ML \& AI & pandas, Tensorflow, scikit-learn, MLFlow,
XGBoost, pola.rs, UMAP, Pytorch, DTW\\
    Databases & PostgreSQL, MySQL, DuckDB\\
    Cloud technologies & Kubernetes, Airflow, Spark\\
    Tools & git, JIRA, Emacs, \LaTeX, pandoc\\
  \end{tabularx}


\end{document}
